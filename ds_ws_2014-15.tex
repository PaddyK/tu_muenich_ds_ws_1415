\documentclass[10pt,a4paper,oneside]{report}
\usepackage[utf8]{inputenc}
\usepackage[german]{babel}
\usepackage{amsmath}
\usepackage{amsfonts}
\usepackage{amssymb}
\usepackage{MnSymbol}
\usepackage{fancyhdr}
\author{Patrick Kalmbach}
\newcommand{\pdfauthor}{Patrick Kalmbach}
\newcommand{\matrikelnummer}{Mtrk-Nr: 03659077}
\newcommand{\fach}{Diskrete Strukturen}
\newcommand{\semester}{Wintersemester 2014/2015}
\newcommand{\doctype}{Homework 1}
\newcommand{\finishdate}{17.10.2014}

\pagestyle{fancy}
\setlength{\headheight}{2.5em}
\lhead{\pdfauthor\linebreak\matrikelnummer}
\rhead{\fach\linebreak\semester}
\chead{\doctype\linebreak\finishdate}
\lfoot{}
\rfoot{}
\cfoot{\thepage}


\begin{document}
\noindent
\section{Homework 1}
\subsection{Exercise 1}
\begin{enumerate}
\item Der Minuend ist der Ausdruck, der vermindert wird $\rightarrow (x+y)$ is Minuend
\item Gültige arithmetische Ausdrücke:
	\begin{itemize}
		\item $x+2$
		\item $+1+2x$ Dieser Ausdruck ist nur dann gültig, wenn es sich bei dem $+$ um ein Vorzeichen und nicht um eine Operation handelt
		\item $(-12x)\cdot x$
		\item $x-y+2$
	\end{itemize}
\item
	\begin{itemize}
		\item $(2+3):(4-4)$ Der Ausduck ist zwar ein Quotient, jedoch ungültig da Division durch Null
	\end{itemize}
	\item Unteschiedliche Auswertungsvorschriften des Ausdrucks $2+3:4-4$:
	\begin{itemize}	
		\item $2+3:4-4=-1.25$
		\item $(2+3):4-4 = -2.75$
		\item $2+3:(4-4) = undef$
		\item $(2+3):(4-4) = undef$
		\item $(2+3:4)-4 = -1.25$
		\item $2+(3:4-4) = -1.25$
	\end{itemize}
	\item Folgendes sind (arithmetische) (Teil-) Ausdrücke:
		\begin{itemize}
			\item Differenz
			\item Multiplikator
			\item Divident
			\item Bruch
		\end{itemize}
\end{enumerate}

\subsection{Exercise 2}
Menge aller geraden Zahlen: $A:=\lbrace 2n\mid n\in\mathbb{Z}\rbrace$

Menge aller ungeraden Zahlen: $B:=\lbrace 2n\mid n\in\mathbb{Z}\rbrace$

Assuming: $\exists x\in A \wedge\in B$ then:
\begin{equation*}
\begin{split}
	2a = 2b+1\qquad\mid :2 \\
	a = b+\frac{1}{2}\not\in\mathbb{Z}\quad\lightning
\end{split}
\end{equation*}

\subsection{Exercise 3}
\begin{enumerate}
\item Die Aussage stimmt nicht:

Sei $x:=33\rightarrow y=13$

$T(x) = \lbrace 11,3,33,1\rbrace$

$T(y) = \lbrace 1,13\rbrace$

$T(20) = \lbrace 1,2,4,5,10\rbrace$

$gT(x,y) = \lbrace 1,13\rbrace \nsupseteq T(20)\quad\lightning$

\item Sei $t\cdot a =x,\quad t\cdot b=20$, then:
\begin{align*}
y&=x-20 \qquad\mid x=t\cdot a,20=t\cdot b	\\
y&=t\cdot a-t\cdot b							\\
y&=t(a-b)									\\
\frac{y}{t}&=(a-b)\qquad
\end{align*}

\item Sei $P(n)$ die Menge der Primzahlfaktoren von n

$P(660)=\lbrace 2,2,3,5,11\rbrace$

$P(470)=\lbrace 2,5,47\rbrace$

$ggT(660,470) = 10,\quad T(660,470)=\lbrace 1,2,5,10\rbrace$
\end{enumerate}

\subsection{Exercise 4}
\begin{enumerate}
	\item Gegeben: $A=\lbrace\epsilon,10,bc,100\rbrace$,$B=\lbrace 10,bc,2,3,\delta\rbrace$, $C=\lbrace a,2,3,10\rbrace$
	
	$D=B\cup(A\cap C)=B\cup\lbrace 10\rbrace=B$
	
	$E=D\cap(A\cup B) = B$
	
	$F=E\cup(A\setminus C)=B\cup\lbrace \epsilon,bc,100\rbrace = \lbrace 10,bc,2,3,\delta,\epsilon,100 \rbrace$
	
	$G=F\cap(A\cup C) = F\cap\lbrace \epsilon,10,bc,100,\alpha,2,3\rbrace = \lbrace 10,\epsilon,bc,100,2,3\rbrace$
	\item $\mid P(\lbrace\emptyset,\lbrace\emptyset\rbrace\rbrace)\mid = \mid\lbrace\emptyset ,\lbrace\emptyset\rbrace ,\lbrace\lbrace\emptyset\rbrace\rbrace ,\lbrace\emptyset ,\lbrace\emptyset\rbrace\rbrace\rbrace\mid = 4$
\end{enumerate}
\end{document}